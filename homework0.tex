\documentclass[12pt]{article}

\usepackage{color}
%\input{rgb}
%----------Packages----------
\usepackage{amsmath}
\usepackage{amssymb}
\usepackage{amsthm}
\usepackage{amsrefs}
\usepackage{dsfont}
\usepackage{enumerate}
\usepackage{hyperref}
\usepackage{mathrsfs}
\usepackage{stmaryrd}
\usepackage{tikz}
	\usetikzlibrary{matrix}
\usepackage[all]{xy}
\usepackage[mathcal]{eucal}
\usepackage{verbatim}  %%includes comment environment
\usepackage{fullpage}  %%smaller margins
%----------Commands----------

%%penalizes orphans
\clubpenalty=9999
\widowpenalty=9999


%% blackboard bold math capitals
\newcommand{\bbA}{\mathbb{A}}
\newcommand{\bbB}{\mathbb{B}}
\newcommand{\bbC}{\mathbb{C}}
\newcommand{\bbD}{\mathbb{D}}
\newcommand{\bbE}{\mathbb{E}}
\newcommand{\bbF}{\mathbb{F}}
\newcommand{\bbG}{\mathbb{G}}
\newcommand{\bbH}{\mathbb{H}}
\newcommand{\bbI}{\mathbb{I}}
\newcommand{\bbJ}{\mathbb{J}}
\newcommand{\bbK}{\mathbb{K}}
\newcommand{\bbL}{\mathbb{L}}
\newcommand{\bbM}{\mathbb{M}}
\newcommand{\bbN}{\mathbb{N}}
\newcommand{\bbO}{\mathbb{O}}
\newcommand{\bbP}{\mathbb{P}}
\newcommand{\bbQ}{\mathbb{Q}}
\newcommand{\bbR}{\mathbb{R}}
\newcommand{\bbS}{\mathbb{S}}
\newcommand{\bbT}{\mathbb{T}}
\newcommand{\bbU}{\mathbb{U}}
\newcommand{\bbV}{\mathbb{V}}
\newcommand{\bbW}{\mathbb{W}}
\newcommand{\bbX}{\mathbb{X}}
\newcommand{\bbY}{\mathbb{Y}}
\newcommand{\bbZ}{\mathbb{Z}}


\renewcommand{\phi}{\varphi}

\renewcommand{\emptyset}{\O}

\providecommand{\abs}[1]{\lvert #1 \rvert}
\providecommand{\norm}[1]{\lVert #1 \rVert}


\providecommand{\ar}{\rightarrow}
\providecommand{\arr}{\longrightarrow}

\renewcommand{\_}[1]{\underline{ #1 }}


\DeclareMathOperator{\ext}{ext}



%----------Theorems----------

\newtheorem{theorem}{Theorem}[section]
\newtheorem{proposition}[theorem]{Proposition}
\newtheorem{lemma}[theorem]{Lemma}
\newtheorem{corollary}[theorem]{Corollary}


\newtheorem{axiom}{Axiom}


\theoremstyle{definition}
\newtheorem{definition}[theorem]{Definition}
\newtheorem{exercise}[theorem]{Exercise}
\newtheorem{example}[theorem]{Example}
\newtheorem{remark}[theorem]{Remark}
\newtheorem{notation}[theorem]{Notation}
\newtheorem{warning}[theorem]{Warning}


\numberwithin{equation}{subsection}


%----------Title-------------

\title{Homework 0}
\author{David Han}
\date{January 2022}

\begin{document}

\maketitle

\section{Conditional probability}
\begin{gather*}
\text{Given $\Omega = \{(r \succ b \succ g), (r \succ g \succ b), (b \succ r \succ g), (b \succ g \succ r), (g \succ b \succ r), (g \succ r \succ b)\}$} \\
\text{where B is $r \succ g$ and A is $b \succ r$} \\ \\
\bbP(B|A)=\frac{\bbP(A\cap B)}{\bbP(A)} \\
\bbP(r \succ g | b \succ r)=\frac{\bbP(b \succ r \cap r \succ g)}{\bbP(b \succ r)}
\end{gather*}

As $\bbP(b \succ r \cap r \succ g)$ is the probability of b being preferred over r and r being preferred over g or the probability of b being preferred over r being preferred over g, it is the possibility of event $(b \succ r \succ g)$ or $1/6$. \\

Likewise $\bbP(b \succ r)$ is the probability of b being preferred over r, meaning it is the probability of events $\{(b \succ r \succ g), (b \succ g \succ r), (g \succ b \succ r)\}$ or $3/6$. \\

\begin{center}
Thus the probability that in this second round the child will choose to play with the toy that we already offered in the first round is: \\
$\dfrac{\bbP(b \succ r \cap r \succ g)}{\bbP(b \succ r)}=\dfrac{1/6}{3/6}=\dfrac{1}{3}$
\end{center}

\section{Probability density functions}
\subsection{Calculate expectation of bid}
\begin{align*}
E[v_J]&=\int_{-\infty}^{\infty}x\cdot f(x)dx \\
&=\int_{1}^{3}x\cdot \frac{1}{2}dx \\
&=\frac{1}{2} \left[ \frac{x^2}{2} \right]_{1}^{3} \\
&=2 \\ \\
E[b_J]&=\int_{1}^{2}x\cdot \frac{1}{2}dx \\
&=\frac{1}{2} \left[ \frac{x^2}{2} \right]_{1}^{2} \\
&=\frac{3}{4}
\end{align*}

\subsection{Calculate probability}
\subsection*{Solution to first subproblem}
\begin{align*}
E[b]&=\int_{-\infty}^{\infty}x\cdot f(x)dx \\
&=\int_{0}^{3}x\cdot \frac{1}{3}dx \\
&=\frac{1}{3} \left[ \frac{x^2}{2} \right]_{0}^{3} \\
&=\frac{3}{2} \\ \\
F(x)&=\int_{-\infty}^{3}f(x)dx \\
&=\int_{-\infty}^{3}\frac{1}{3}dx
\end{align*}

\subsection*{Solution to second subproblem}
\begin{align*}
\bbP(b\leq 2)&=\int_{0}^{2}x\cdot \frac{1}{3}dx \\
&=\frac{1}{3} \left[ \frac{x^2}{2} \right]_{0}^{2} \\
&=\frac{2}{3} \\
\end{align*}

\subsection*{Solution to third subproblem}
\begin{align*}
\bbP \left( b \leq \frac{3}{4} \right)&=\int_{0}^{3/4}x\cdot \frac{1}{3}dx \\
&=\frac{1}{3} \left[ \frac{x^2}{2} \right]_{0}^{3/4} \\
&=\frac{3}{32} \\
\end{align*}

\section{Combinatorics}
\subsection*{Solution to first subproblem}
\[
\bbP(\text{No one})=\frac{1}{10!}
\]

\subsection*{Solution to second subproblem}
Since there is only 9 possible students the first student can switch with, then 8 possible students the second student can switch with, and so forth to 1.
\[
\bbP(\text{Exactly one student})=\frac{45}{10!}
\]

\subsection*{Solution to third subproblem}
Since two swaps will be needed only if there are 3 out of place students in a chain or 4 out of place students in two separate chains, we can determine the probability of two swaps being needed by determining the probability of these outcomes. \\

% For 4 out of place students in two separate chains, there is  
\[
\bbP(\text{Exactly two students})=\frac{870}{10!}
\]

\end{document}
